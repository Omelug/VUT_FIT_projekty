
\documentclass{article}
\usepackage[simplified]{pgf-umlcd}4
\usepackage[legalpaper, landscape, margin=2in]{geometry}
\usepackage{tikz}


\newcommand{\pk}[1]{\attribute{\textless\textless PK\textgreater\textgreater\ #1}} %primary key

\begin{document}
    TODO    \\
    Er diagram \\ \\

    4: Fake news \\
    Konzultant: ing. Dacík\\
    Kancelář pro uvádění románových příběhů na pravou míru (DOOM) se rozhodla rozšířit svoji působnost do on-line prostoru, a požádala Vás proto o návrh ER diagramu, který by byl základem informačního systému pro vyhodnocování tzv. fake news (IS CANE). Sada botů bude procházet internet a nalezené zprávy bude ukládat do IS CANE. Jednotlivé zprávy budou identifikovány svým URL a dále bude zaznamenáno datum a čas jejich zveřejnění a osoba, která zprávu zveřejnila (pokud ji lze zjistit). Aby bylo možné pracovat i se zprávami, které jsou později smazány, systém si celou zprávu uloží (v IS CANE bude uvedeno datum a čas uložení zprávy a cesta k její kopii).
Nalezené zprávy budou postupně vyhodnocovány týmem DOOM.

    Protože se ale prakticky stejná zpráva obvykle vyskytuje na mnoha místech, vyhodnocení bude probíhat najednou pro celou skupinu těchto zpráv.\\
    Vybraný člen vypracuje na zprávu (přesněji na skupinu zpráv) posudek, ve kterém navrhne hodnocení (pravdivá, nepravdivá, zavádějící, neověřitelná) a důkladně ho odůvodní, a to včetně uvedení použitých zdrojů (konkrétních zpráv) a případného upřesnění, jak byla která zpráva pro posudek využita.\\
    Na základě posudku vedoucí DOOM přidělí zprávě hodnocení nebo, pokud není s některým posudkem spokojen, může pověřit dalšího člena vypracováním nového posudku. Všechny posudky budou v IS CANE evidovány (je třeba znát pořadí posudku pro danou skupinu zpráv, datum a čas vypracování a autora).\\
    Pro snadnější orientaci veřejnosti v hodnocených zprávách přidělí vedoucí DOOM skupině zpráv výstižný (unikátní) název, vybere nejčastější znění zprávy, zaeviduje datum a čas jejího prvního výskytu a shrne odůvodnění hodnocení. Zprávy také zařadí do oblastí (oblast je charakterizována svým názvem a popisem), přičemž návrh na zařazení do jednotlivých oblastí (jedna zpráva může spadat i do více oblastí) bude již součástí posudků.\\
Osobám, ať už publikují na internetu nebo píší posudky, IS CANE vygeneruje osobní číslo a dále bude uchovávat (když budou tyto údaje známy) jejich pseudonym, jméno a příjmení, e-mail, telefon, případně facebookový a twitterový účet.\\
    \newpage
    \begin{tikzpicture}
        \begin{class}[text width=6cm]{Bot}{3,5}
                \attribute {ID}
            \end{class}
        \begin{package}{IS CANE}
            \begin{class}[text width=6cm]{Zpráva}{-2,0}
                \pk{URL}
                \attribute {Datum a čas uložení}
                \attribute {Autor}

                \attribute {Název}
                 \attribute {Datum a čas zveřejnění}
                \attribute {nehodnocen/nespokojen/hodnocena}
            \end{class}

            \begin{class}[text width =4cm]{Skupina zpráv}{5, 0}
            \attribute {Text}
            \end{class}

            \begin{class}[text width =3cm]{Posudek}{10,0}
                \attribute{pořadí}
                \atribute{pravdivá, nepravdivá, zavádějící, neověřitelná)}
                \attribute{odůvodnění}
            \end{class}

            \begin{class}[text width =4cm]{DOOM Člen}{0,-10}
            \end{class}

            \begin{class}[text width =2cm]{Vedouci DOOM}{5,-10}
                \inherit {DOOM Člen}
            \end{class}

            \begin{class}[text width =4cm]{Uživatel}{5,-5}
            \end{class}
        \end{package}
        \draw [umlcd style,fill=none,->] (Zpráva) -| (Skupina zpráv)
        \draw [umlcd style,fill=none,->] (Bot) -| (Zpráva)
        \node at (-0.5,2) {Uložení zprávy};
        % propojeni
        %\association{Zpráva}{*}{}{Skupina zpráv}{1}
        \unidirectionalAssociation{Skupina zpráv}{}{1..1}{Posudek}
    \end{tikzpicture}
\end{document}

